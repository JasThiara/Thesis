\chapter{Asset Bubble}
\begin{enumerate}
  \item Market forces exhibit stochastic behavior.
  \begin{itemize}
    \item Stochastic behavior operates in a probabilistic sense\cite{KSFM}. 
    \item Uncertainties in market value \cite{FSPT}.
    \item Intrinsic value type 1, 2 and 3 bubbles\cite{GSA}.
    \item Extrinsic value(bubble)\cite{JAPBIIM,JAPBICM}.
    \item Uncertainties(risks) in financial portfolios, and sectors of markets(e.g .com bubble, housing bubble)\cit{JAPBIIM}
  \end{itemize}

  \item To minimize risk in portfolio against bubble, one would need to detect when a bubble exists in an asset\cite{FZOETD}.
  \item In this paper, we use a non-parametric estimator Floren Zmirou to detect or determine type 3 bubbles in an asset.
  \item Further testing is needed to determine if this method is valuable in minimizing portfolio risk against type 3 bubble.
\end{enumerate}



\section{Introduction}
\indent Financial asset bubbles are a trending topic linked with enthusiasm and interest in today's economy.
In keeping a watchful eye on recent market trends, it has been noted that there has been an escalation in the prices of gold.
Financial entrepreneurs and other interested parties are maintaining their posture, curios as to the future changes in these market prices.
There are uncertainties in market value of an asset.The stock market is unpredictable, unstable and an insecure source to grow one's net worth however; despite posing a risk, the stock market still provides an avenue to increase one's net worth and increase value in assets.
The Divarication in parameter decisions ultimately decides if the regression capacity over�ts, under�ts, or �ts ideally.
The association between characteristic spaces and smoothness is not self-evident, and subsequently, must be inspected utilizing a various methodologies.
Characteristic feature spaces may further be utilized to compare or synchronize articles which have a much greater complex arrangement.\\
\indent How are we able to detect or estimate the future changes of any asset, such as stock, gold, housing, or various commodities? 
How quickly will an asset price will jump, or fall? These are the questions which we will maintain our focus throughout this study, as we center our approach and define our reasonings.
To minimize the risk in  portfolio against bubble, one would need to detect when a bubble exists in an asset.\\
\indent  Using some helpful techniques to determine an asset bubble in real time will help financial corporations, banks, and money markets.
They can lower their money damaging risks by using our methodology. According to ``There is a bubble'' paper paragraph 2, indeed, in 2009 the federal reserve chairman Ben bernanke said in congress testimony[1] ``It is extraordinarily difficult in real time to know if an asset price is appropriate or not''.
There are three-types of asset price bubbles exist in finance market. Type 1 and type 2 bubbles exist in infinite horizon economies but type 3 bubbles exist in finite horizon.
Our focus is to detect type 3 bubble. 
The process for bubble detection depends on a mathematical analysis that determines when an asset is undergoing speculative pricing.\\
Our method will consit of historical stock prices , estimating its volatility and then interpolation the volatility.\\ 
\indent We will estimate volatility $\sigma(x)$ by  non-parametric estimator Florens Zmirou and Interpolate the volatility by cubic spline which will determine whether any asset is experiencing a price bubble in finite time interval.\\
As stated above, we will use a nonparametric estimator Florens -Zmirou which is based on the local time of the diffusion process.
The biggest challenge we have faced in using a non-parametric estimator, is that we can only estimate $\sigma(x)$ volatility function on the points which are visited by the process.
Only a finite number of data points are used which is a compact subset of $R+$, which will help us to detect the behavior of function. 
\indent After interpolation of volatility function $\sigma(x)$, one need to extrapolate the volatility function to check if the intergral based on volatility is finite or infinite.
If the integral is finite , stock price has a bubble and by doing so we can determine whether the price process under a risk neutral measure is a martingale or strict local martingale.
As lack of resources and time, we are not able to estimate the tail of the volatility and we do not, as of yet, know the asymptotic behavior of the volatility.
Further testing is needed to determine if this method is valuable in minimizing portfolio risk against type 3 bubble.\\
\indent My thesis work will be organized as follows: Next section will show the previous work on detection on asset price bubble. 
In chapter 2, we will present objective, background, and  mathematical background related to portfolio theory.
We will describe the connection between finance, mathematics terms and their definations.
In chapter 3, we will discuss the details of our algorithm, and we will further examine the probable and the extent of its implementation.
In chapter 4, we will present several numerical examples and variables. We will finally present our conclusion and future work. \\

\section{Previous Work}
\subsection{Variance Bounds tests of an Asset Priceing Model}
In 1986, Robert P. Flood and Robert J. Hodrick Introduced the methods which examine whether some of the variance bounds tests reported to data
provide evidence for the hypotesis that asset price contain speculative bubbles and malformed time series as a method of detection and
detection difficulty respectively \cite{FAPVBPS}. 
\subsection{A panel Data approach} In 2007, A panel data test and cointegration are used to detect bubbles
in housing prices.\cite{MTFBIHM} 
\subsection{Bootstrape Method} in 2010 Luciano Gutierrez used a bootsrapping
methodology to indicate type 3 bubbles in the Nasdaq price index and Case-Shiller house price index.
The method is mainly based on a the finite sample probability distribution of the asymptotic\cite{GBAPB}.
\subsection{State Space Model with Markov Switching}In 2011, Guojin Chen and Cheng Yan introduced a Kalman filter to estimate a bubble and then use a state space model with Markov switching
to model the survivability of a bubble, switching between collapsing and surving states\cite{YAFEAISMB} .
\subsection{Reproducing Kernal Hilbert Space} In 2011, Robert A. Jarrow, Younes Kchia,and Philip Protter used to RKHS theory to detect asset bubbles\cite{bigBubble}.