\chapter{Introduction}
%Explain the background of asset bubbles, risk, portfolio theory
\section{Previous Work on Asset Bubble Detection}


% \chapter{Bubbles}
% \section{Introduction}
% \indent In today's economy, financial asset bubbles are a trending topic linked with enthusiasm and interest. 
% In keeping a watchful eye on recent market trends, it has been noted that there has been an escalation in the prices of Gold. 
% Financial entrepreneurs and other interested parties are maintaining their posture, curios as to the future changes in these market prices. 
% How are we able to detect or estimate the future changes of any asset, such as stock, gold, housing, or various commodities? How quickly will an asset price will jump, or fall? 
% These are the questions which we will maintain our focus throughout this study, as we center our approach and define our reasonings. 
% If a computer program existed that accurately estimated the stock market and gave a real time minute by minute update, this would benefit not only day traders and investors but this tool would increase the quality of life in every dynamic for anyone who consistently made a fortune in the stock market. 
% In this paper we will study non-parametric estimator done by Florens Zmirou and Interpolation by cubic spline. 
% Therefore, we will reveal how to determine whether any asset is experiencing a price bubble in real time.\\
% \indent How will we detect asset bubbles in real time? 
% Our research problem will be in deciding whether an asset price is experiencing a price bubble in a finite, or an infinite time period.\\
% \indent We will be able to determine the volatility of asset prices. Using some helpful techniques to determine an asset bubble in real time will help financial corporations, banks, and money markets. 
% They can lower their money damaging risks by using our methodology. According to ``There is a bubble'' paper paragraph 2, indeed, in 2009 the federal reserve chairman Ben bernanke said in congress testimony[1] ``It is extraordinarily difficult in real time to know if an asset price is appropriate or not''. 
% Our goal is to estimate stock price volatility by way of the Florens Zmirou estimator, and then we will extrapolate the volatility tail in order to check the integral. 
% We will then be able to determine whether the integral is finite or infinite. The process for bubble detection depends on a mathematical analysis that determines when an asset is undergoing speculative pricing, i.e., its market price is greater than its fundamental price. 
% The difference between the market and the fundamental price, is a price bubble.\\
% \indent As stated above, we will use a nonparametric estimator Florens -Zmirou which is based on the local time of the diffusion process. 
% The biggest challenge we have faced in using a non-parametric estimator, is that we can only estimate $\sigma(x)$ volatility function on the points which are visited by the process. 
% Only a finite number of data points are used which is a compact subset of R+, therefore, we are not able to estimate the tail of the volatility. 
% However, by determining the tail of the volatility, we can then reveal whether the integral is finite or infinite. 
% We do not, as of yet, know the asymptotic behavior of the volatility.
% Only finite data points are used which is a compact subset of $R+$.  /therefore we will not able to estimate the tail of volatility.\\
% \indent After an estimation of the volatility function $\sigma(x)$, we will then interpolate the function using cubic splines and Reproducing Kernel Hilbert Spaces. 
% Since a Hilbert Space is an inward item space that is finished and distinct regarding the standard de�ned by the internal item. 
% The stock market is unpredictable, unstable and an insecure source to grow one's net worth however; despite posing a risk, the stock market still provides an avenue to increase one's net worth and increase value in assets.  
% My program once debugged and completed will accurately estimate the volatility of stock and will interpolate the volatility function.
% We estimate the the volatility of asset prices using the Floren Zmirou estimator and then we will extrapolate the volatility tale in order to test if the prices exhibits a martin angle or submartinangle.  
% If the volatility is large enough then a bubble exists.  
% A non-parametric estimator Florens Zmirou, which is based on local tine of the diffusion process to estimate the volatility, function on the points which are visited by the process. 
% The Divarication in parameter decisions ultimately decides if the regression capacity over�ts, under�ts, or �ts ideally. 
% The association between characteristic spaces and smoothness is not self-evident, and subsequently, must be inspected utilizing a various methodologies. 
% Characteristic feature spaces may further be utilized to compare or synchronize articles which have a much greater complex arrangement.\\
% \indent In chapter 2, we will present an overview of previous work, literature reviews, and alternate methodologies and their outcome measures. 
% We will then present the background of the problem, and we will present how the problem is connected to finance and mathematics. 
% We will further present the methods we have uncovered to solve the problem, complete with our best possible solution to the problem. 
% In chapter 3, we will discuss the details of our algorithm, and we will further examine the probable and the extent of its implementation. 
% In chapter 4, we will present several numerical examples and variables. We will finally present our conclusion and future work.\\
% 
% \section{Previous Work}
% There are many methods which were reviewed by many researchers Since 1986.
% \subsection{Variance Bounds Tests of an Asset Pricing Model}
% The `` Asset Price Volatility, Bubbles, and Process Switching'' artile was published in 1986. This test will show that bubbles could in theory lead to excess volatility,but it shows that certain variance bounds tests preclude bubbles as an explanation.
% \subsection{A Panal Data Approach}
% In 2007,''Testing for Bubbles in Housing Markets'' Vyacheslav Mikhed and Petrz Zemcik developed cross-sectionally robust panel data tests for unit roots and cointegration to find whether house prices reflect house related earnings.
% \subsection{Reproducing Kernal Hilbert Space}
%  In 2011,''How to detect Asset Bubble'' by Jarrow Robert, Kchia Younes, and Protter Philip used Reproducing kernal Hilbert Space and Optimization to detect asset Bubbles.
% \subsection{A monte Carlo Simulation}
% In 2011, ``Bootstrapping Asset Price Bubbles'' by Luciano Gutierrez used Monte Carlo Simulation.bootstrap procedures often provide better finite sample critical values for test-statistics than asymptotic theory does, bootstrap values are still approximations and are not exact. Davidson and Mackinnon (1999, 2006)
% show that it is possible to estimate the size and power of bootstrap tests by running Monte Carlo simulations, when the condition of asymptotic independence of the bootstrapped statistic and the bootstrap Data Generating Process (DGP) holds.
% \subsection{State Space Model With Markov-Switching}
% In 2011,  ``A Financial Engineering Approach to Identify Stock Market Bubble'' by Guojin Chen and Chen Yan adopt a state space model with Markov-switching to identify the stock market bubbles both in China and US.
% % 
